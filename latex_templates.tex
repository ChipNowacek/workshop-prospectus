
\documentclass[12pt]{article}
\usepackage{graphicx}
\usepackage{hyperref}
\usepackage{glossaries}

% Create glossary
\makeglossaries

% Example Glossary Entries
\newglossaryentry{cadre}{
    name={Cadre},
    description={A small group of people specially trained for a particular purpose or profession. Used here to mean the founding group of professionals stewarding family wealth transitions.}
}

\begin{document}

\section*{Templates}

\subsection*{Bibliography Entry (APA Style)}
\begin{verbatim}
@book{chernow1998titan,
  author    = {Chernow, Ron},
  title     = {Titan: The Life of John D. Rockefeller, Sr.},
  year      = {1998},
  publisher = {Random House},
}

@article{smith2020wealth,
  author  = {Smith, J. A. and Doe, R. B.},
  title   = {Wealth transition strategies in family businesses},
  journal = {Journal of Family Enterprise},
  volume  = {45},
  number  = {3},
  pages   = {201--218},
  year    = {2020},
  doi     = {10.xxxx/xxxxx}
}
\end{verbatim}

\subsection*{Figure Entry Template}
\begin{verbatim}
% In the main document where the figure appears:
\begin{figure}[h]
    \centering
    \includegraphics[width=0.7\textwidth]{figure-file.png}
    \caption[Short Caption for List of Figures]{Title – Detailed description explaining the figure’s content, context, and relevance.}
    \label{fig:ecosystem-model}
\end{figure}
\end{verbatim}

\subsection*{Glossary Entry Template}
\begin{verbatim}
% Preamble:
\usepackage{glossaries}
\makeglossaries

% Entry:
\newglossaryentry{term}{
    name={Term},
    description={Concise, clear explanation with optional context or notes.}
}

% Use in text:
\gls{term}
\end{verbatim}

\printglossaries

\end{document}
